\mychapter{Metodologia}{cap:metodologia}

A metodologia da pesquisa desenvolvida consistiu em dois grandes momentos, a etapa teórica e a implementação das funcionalidades.

Para o desenvolvimento do projeto descrito neste trabalho, foram planejadas as seguintes etapas de execução, que foram realizadas ao longo dos últimos meses.

\section{Estudo das funções a serem implementadas}

O estudo das funções a serem implementadas foi realizado a partir da análise de um conjunto de referências bibliográficas de qualidade, visando ampliar os conhecimentos a cerca do tema proposto.

Foram estudados ao longo deste momento, temas como séries temporais, suas propriedades e aplicações, Teoria da Informação, entropias~\cite{salicruetal1993}, distâncias estocásticas~\cite{StatisticalInferenceBasedonDivergenceMeasures}, complexidades estatísticas, plano HC e a linguagem de programação \texttt R.

\section{Implementação e validação numérica}

Após o término da revisão bibliográfica da literatura existente, foi dado então início à implementação do trabalho, desenvolvido em \texttt R e sempre fazendo uso de boas práticas de desenvolvimento de software científico.

Para que tal ferramenta seja aplicada na análise de dados é de suma importância realizar a verificação de suas propriedades numéricas. 
Portanto, a avaliação da qualidade numérica das funcionalidades desenvolvidas foi feita utilizando uma metodologia própria baseada em sistemas dinâmicos com saídas conhecidas.

\section{Análise de alternativas para o desenvolvimento da interface}

Um dos grandes objetivos da pesquisa consistia em ampliar a aplicabilidade das técnicas de extração de informações de séries temporais, por meio de uma ferramenta portável e interativa de análise. Assim, foram avaliadas algumas opções de ferramentas de GUI que fossem capaz de suportar as funcionalidades desenvolvidas em \texttt R na primeira etapa.

Foi então realizada uma pesquisa sobre as alternativas existentes sendo considerado os seguintes fatores:

\begin{itemize}
\item Portabilidade do software para os diversos sistemas operacionais e arquiteturas de hardware; 
\item Facilidade de instalação, pois uma vez que queremos por meio do desenvolvimento do projeto facilitar de um modo geral a análise de séries temporais na experiência do usuário, logo esta não deverá apresentar problemas no processo de instalação;
\item Integração com a linguagem de programação \texttt R.
\end{itemize}

Desse modo, \texttt{RGtk2} e \texttt{Java Swing} foram as alternativas iniciais para o desenvolvimento da interface gráfica. 
No entanto, após estudos sobre o funcionamento destas GUIs (\textit{Graphical User Interface}), verificamos que a implementação da interface utilizando \texttt{Java Swing} apresentava certos empecilhos em relação a portabilidade do software em diferentes sistemas operacionais, não satisfazendo ao item 1 de nossas exigências, seria necessário a implementação individual do software para cada sistema operacional, já que o programa deveria ser capaz de  reconhecer o sistema utilizado pelo cliente e assim executar seguindo as regras e padrões deste. Outro fator decisivo foram as dificuldades de comunicação entre o código \texttt{Java} e o script em \texttt{R}.

Portanto, optamos pelo \texttt{RGtk2}, por ser uma biblioteca própria do ambiente de desenvolvimento \texttt R e pela sua maior facilidade em manter a portabilidade do sistema.

\section{Desenvolvimento de protótipos}
 
Foram desenvolvidos alguns protótipos de modelos de interface com as alternativas de bibliotecas gráficas citadas anteriormente, sempre com foco na experiência do usuário. 

No entanto, por possuímos como objetivo o desenvolvimento de uma ferramenta \texttt{Desktop} algumas alterações foram realizadas para se adequar as funções oferecidas pela biblioteca escolhida.

\section{Versão de produção da interface}

Após a finalização do processo de escolha da biblioteca \texttt{RGtk2}, foi então dado início a implementação da interface.
Esta etapa consistiu basicamente da realizada da integração entre o ambiente gráfico do sistema e as funções de análise de séries temporais implementadas em fases anteriores.

\section{Validação, verificação e preparação de manuais e tutoriais de uso}

Como já citado, é de fundamental importância para tal projeto a verificação da qualidade numérica do software desenvolvido, portanto um dos seus objetivos consistiu em validar a interface e as funções com usuários finais.

Foram também desenvolvidos manuais de uso das funções implementadas, informando as suas funcionalidades, parâmetros de entrada e o resultado final computado. Todas essas descrições se encontram apresentados no apêndice A deste trabalho.
