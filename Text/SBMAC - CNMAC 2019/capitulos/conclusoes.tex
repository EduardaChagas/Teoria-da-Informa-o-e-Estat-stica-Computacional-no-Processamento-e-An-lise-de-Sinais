\mychapter{Conclusões}{cap:conclusoes}

Neste capítulo serão abordados os avanços no meio científico e a importância proporcionada através do desenvolvimento deste trabalho. Além disso, também apresentaremos sugestões para futuros trabalhos.

\section{Considerações Finais}

Este trabalho propôs o desenvolvimento de uma ferramenta portável, rápida e de boa qualidade numérica que possibilita análises de uma série temporal através de descritores provenientes da Teoria da Informação. Para atribuir uma função de distribuição de probabilidade utilizamos o método de simbolização de Bandt-Pompe. A caracterização dos dados é dada por meio dos seus descritores, sendo então disponibilizadas diversas entropias, distâncias estocásticas e complexidade estatística.

Um elemento original do sistema é a vinculação entre o histograma de padrões e a série temporal. Escolhendo um ou mais elementos do histograma, os valores correspondentes na série temporal aparecem realçados. Esta funcionalidade permite a análise visual da distribuição temporal dos padrões, possibilitando futuramente a realização de outros testes.

O projeto também oferece aos pesquisadores a facilidade de utilização de técnicas sofisticadas da computação científica por meio de uma interface simples e intuitiva, sendo possível realizar em poucos passos atividades antes realizadas apenas por meio de scripts, exigindo assim mínimo conhecimento com programação por parte do usuário.

\section{Trabalhos futuros}

Pretendemos expandir as funcionalidades do sistema, dando agora ênfase ao problema da imputação de padrões ausentes. Para tanto, pretendemos atingir os seguintes objetivos:

\begin{itemize}
\item Estudar e implementar técnicas para imputação de padrões ausentes ocasionados por dados repetidos;
\item Analisar a capacidade de reconstrução de informações dessas técnicas quando a série temporal é armazenada com menos precisão do que a ideal;
\item Analisar a distribuição temporal dos padrões originais e imputados.
\end{itemize}