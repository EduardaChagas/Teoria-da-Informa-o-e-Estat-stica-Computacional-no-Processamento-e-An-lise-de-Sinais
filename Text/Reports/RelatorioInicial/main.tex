\documentclass[a4paper]{article}

%% Language and font encodings
%%\usepackage[english]{babel}
\usepackage[portuguese]{babel}
\usepackage[utf8x]{inputenc}
\usepackage[T1]{fontenc}
\usepackage{cite}


%% Sets page size and margins
\usepackage[a4paper,top=3cm,bottom=2cm,left=3cm,right=3cm,marginparwidth=1.75cm]{geometry}

%% Useful packages
\usepackage{amsmath}
\usepackage{graphicx}
\usepackage[colorinlistoftodos]{todonotes}
\usepackage[colorlinks=true, allcolors=blue]
{hyperref}


\title{Relatório parcial de reuniões referentes às primeiras atividades realizadas acerca do projeto PIBIC/PIBITI de 2018/2019.}
\author{Demétrios Reis \\Milena Nunes}

\begin{document}
\maketitle

\section{Reunião 11 de abril}

Nessa primeira reunião fizemos uso do nosso tempo para esclarecermos dúvidas a respeito dos projetos. Questões relativas aos prazos, pontualidade e qualidade de trabalho foram amplamente discutidas pelos membros mais experientes.

Nesta reunião conhecemos a metodologia do grupo e os padrões dos commits que devem ser feitos. Também deixamos claro a diferença e o foco dos trabalhos que iremos executar, pois é crucial distinguir a proposta do PIBIC e PIBITI.

Fizemos algumas ponderações acerca da experiência de pesquisadores que trabalham na linha de pesquisa em questão (A.C Frery, O. Rosso,...) a fim de buscar um tema que fosse aplicável a proposta do PIBITI. 

O aluno Demétrios Reis manifestou o interesse em desenvolver seus trabalhos na área de análise de séries temporais aplicadas ao mercado financeiro. De forma isolada conheceu o alguns papers do professor Osvaldo Rosso e  orientações feitas pelo Prof. Reinaldo Martinez Palhares.

\section{Reunião 16 de abril}

Em nossa segunda reunião, foram-nos apresentadas algumas plataformas que necessitamos conhecer para realizarmos nosso trabalho como o esperado, tais como, \texttt LaTex, \texttt markdown, \texttt ggplot, \texttt RStudio e \texttt GitHub. 

Dessa forma tiramos dúvidas acerca das propriedades dessas plataformas, entendendo melhor a forma como o grupo trabalha e quais as exigências a partir de padrões cobrados tanto nas estruturas de artigos e relatórios, feitos no \texttt LaTex, quanto nos gráficos gerados através do \texttt ggplot. Também tivemos uma breve explicação de como consultar e criar bibliotecas na linguagem \texttt R através do \texttt RStudio. 

Finalizando, também fomos apresentados ao conteúdo que existe no \texttt GitHub desenvolvido pelos bolsistas antes do nosso ingresso no projeto e aprendemos a procurar sites seguros para a busca de referências para o trabalho que viermos a realizar.

\section{Reunião 23 de abril}

Com alguns conceitos praticamente estabelecidos, utilizamos nossa reunião para discutir assuntos referentes ao survey que foi proposto, que pode ser encarado como uma ótima oportunidade para que os membros do grupo possam encarar um desafio que pode elucidar ainda mais os objetos de estudos, fazendo naturalmente uma revisão do que há de mais atual no uso de séries temporais aplicadas na teoria da informação. 

Discutimos métodos, metodologias e divisão de assuntos para o trabalho em questão. Decidimos abordar Distribuições, Entropia, Distâncias Estocásticas, Complexidade e o plano HC. Tivemos um survey como modelo para nos guiarmos nas nossas escolhas referentes a estrutura: ~\cite{survey}.

Após essa reunião tivemos um contato diário e bem mais produtivo uma vez que tínhamos um objetivo com data de entrega marcada. O evento adiou a data de submissão dos trabalhos, fazendo com que tivéssemos mais tempo para nos aprofundar ainda mais nos assuntos. 

Contamos ainda com a apresentação do membro John Omena, onde o mesmo pode, com muita qualidade, reforçar a ideia do que se espera de um projeto PIBITI e como o mesmo pensa em aplicar seus conhecimentos para contribuir com no desenvolvimento do seu projeto.

\bibliographystyle{unsrt}
\bibliography{sample}

\end{document}