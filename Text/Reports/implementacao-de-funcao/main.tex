\documentclass[12pt,letterpaper]{article}
\usepackage[utf8]{inputenc}
\usepackage[spanish, es-tabla]{babel}
\usepackage[version=3]{mhchem}
\usepackage[journal=jacs]{chemstyle}
\usepackage{amsmath}
\usepackage{amsfonts}
\usepackage{amssymb}
\usepackage{makeidx}
\usepackage{xcolor}
\usepackage[stable]{footmisc}
\usepackage[section]{placeins}
\usepackage{listings}
%Paquete para el manejo de las unidades
\usepackage{siunitx}
\sisetup{mode=text, output-decimal-marker = {,}, per-mode = symbol, qualifier-mode = phrase, qualifier-phrase = { de }, list-units = brackets, range-units = brackets, range-phrase = --}
\DeclareSIUnit[number-unit-product = \;] \atmosphere{atm}
\DeclareSIUnit[number-unit-product = \;] \pound{lb}
\DeclareSIUnit[number-unit-product = \;] \inch{"}
\DeclareSIUnit[number-unit-product = \;] \foot{ft}
\DeclareSIUnit[number-unit-product = \;] \yard{yd}
\DeclareSIUnit[number-unit-product = \;] \mile{mi}
\DeclareSIUnit[number-unit-product = \;] \pint{pt}
\DeclareSIUnit[number-unit-product = \;] \quart{qt}
\DeclareSIUnit[number-unit-product = \;] \flounce{fl-oz}
\DeclareSIUnit[number-unit-product = \;] \ounce{oz}
\DeclareSIUnit[number-unit-product = \;] \degreeFahrenheit{\SIUnitSymbolDegree F}
\DeclareSIUnit[number-unit-product = \;] \degreeRankine{\SIUnitSymbolDegree R}
\DeclareSIUnit[number-unit-product = \;] \usgallon{galón}
\DeclareSIUnit[number-unit-product = \;] \uma{uma}
\DeclareSIUnit[number-unit-product = \;] \ppm{ppm}
\DeclareSIUnit[number-unit-product = \;] \eqg{eq-g}
\DeclareSIUnit[number-unit-product = \;] \normal{\eqg\per\liter\of{solución}}
\DeclareSIUnit[number-unit-product = \;] \molal{\mole\per\kilo\gram\of{solvente}}
\usepackage{cancel}
%Paquetes necesarios para imágenes, pies de página, etc.
\usepackage{graphicx}
\usepackage{lmodern}
\usepackage{fancyhdr}
\usepackage[left=4cm,right=2cm,top=3cm,bottom=3cm]{geometry}

%Instrucción para evitar la indentación
%\setlength\parindent{0pt}
%Paquete para incluir la bibliografía
\usepackage[backend=bibtex,style=chem-acs,biblabel=dot]{biblatex}
\addbibresource{references.bib}

%Formato del título de las secciones

\usepackage{titlesec}
\usepackage{enumitem}
\titleformat*{\section}{\bfseries\large}
\titleformat*{\subsection}{\bfseries\normalsize}

%Creación del ambiente anexos
\usepackage{float}
\floatstyle{plaintop}
\newfloat{anexo}{thp}{anx}
\floatname{anexo}{Anexo}
\restylefloat{anexo}
\restylefloat{figure}

%Modificación del formato de los captions
\usepackage[margin=10pt,labelfont=bf]{caption}

%Paquete para incluir comentarios
\usepackage{todonotes}

%Paquete para incluir hipervínculos
\usepackage[colorlinks=true, 
            linkcolor = blue,
            urlcolor  = blue,
            citecolor = black,
            anchorcolor = blue]{hyperref}

%%%%%%%%%%%%%%%%%%%%%%
%Inicio del documento%
%%%%%%%%%%%%%%%%%%%%%%

\begin{document}
\renewcommand{\labelitemi}{$\checkmark$}

\renewcommand{\CancelColor}{\color{red}}

\newcolumntype{L}[1]{>{\raggedright\let\newline\\\arraybackslash}m{#1}}

\newcolumntype{C}[1]{>{\centering\let\newline\\\arraybackslash}m{#1}}

\newcolumntype{R}[1]{>{\raggedleft\let\newline\\\arraybackslash}m{#1}}

\begin{center}
	\textbf{\LARGE{Implementação de funções para análise de dados bidimensionais}}\\
	\vspace{7mm}
	\textbf{\large{LaCCAN - Laboratório de computação científica e análise numérica}}\\ 
	\vspace{4mm}
	\textbf{\large{Aluna: Eduarda Chagas}}\\
	\vspace{4mm}
	\textbf{\large{Orientador: Alejandro Frery}}\\
\end{center}

\vspace{7mm}

\section*{\centering Resumo}

Este relatório possui como principal objetivo informar a lógica de desenvolvimento da implementação da seção de análise de texturas, bem como estas foram estudadas de acordo com a metodologia de Bandt e Pompe. Através deste também viso expor a metodologia empregada e as dúvidas formadas ao longo do processo.

\section{Introdução}

Como demonstrado anteriormente o projeto já possui grande parte de suas funções desenvolvidas, entre estas encontram-se todas aquelas relacionadas a análise de dados unidimensionais e extração dos descritores causais (Entropia, distâncias estocásticas e complexidade estatística).

Logo, para estender a aplicação do software para análise de elementos bidimensionais foram realizadas algumas mudanças no código. Como previsto necessitaram ser implementadas novas funcionalidades e outras foram remodeladas para adequar a nova demanda.

\section{Resultados}

\subsection{Funções de leitura de imagens}

Após uma breve pesquisa a cerca dos mais comuns tipos de imagens, foram implementadas as funções de leitura dos seguintes tipos de dados:  \textit{JPEG}, \textit{PNG}, \textit{BMP} e \textit{TIFF}. Para a realização de tais implementações foram utilizados as seguintes bibliotecas:  \textit{jpeg}, \textit{png},\textit{bmp} e \textit{tiff} \\

\textbf{Input :} O usuário poderá escolher dentre os arquivos de seu computador uma imagem correspondente ao tipo informado pela função.

\textbf{Output :} A imagem selecionada.\\

\begin{lstlisting}
Read_jpeg<-function(){
	myImage<-readJPEG(file.choose())
	return(myImage)
}

Read_png<-function(){
	myImage<-readPNG(file.choose())
	return(myImage)
}

Read_tiff<-function(){
	myImage<-readTIFF(file.choose())
	return(myImage)
}

Read_bmp<-function(){
	myImage<-readBMP(file.choose())
	return(myImage)
}
\end{lstlisting}

\subsection{Função \textit{transformMatrix}}

Sendo utilizada no processo de formação dos símbolos de Bandt e Pompe, tal função possui como finalidade retornar a matriz de níveis de cinza da imagem fornecida como parâmetro.\\

\textbf{Input :} Imagem, dimensão da imagem.

\textbf{Output :} matriz de níveis de cinza.\\

 \begin{lstlisting}
transformMatrix <-function(myImg,dimension){
  myMatrix <- matrix(nrow=dimension,ncol=dimension)
  for(i in 1:dimension){
    for(j in 1:dimension){
      myMatrix[i,j] = (0.3*myImg[i,j,1])+(0.59*myImg[i,j,2])+(0.11*myImg[i,j,3])
    }
  }
  return(myMatrix)
}
\end{lstlisting}

\subsection{Função \textit{formationPatternsImage}}

Assim como sugere o nome da mesma, esta possui como objetivo obter os símbolos de Bandt e Pompe, sendo fornecidos uma determinada imagem, as dimensições e os delays dos futuros padrões. Desse modo , o algoritmo realiza o particionamento da matriz e utilizando a função \textit{order} da linguagem \textit{R} os conjuntos são ordenados originando os símbolos que deveram ser retornados.\\

\textbf{Input :} Imagem, dimensões X e y dos padrões, delays X e Y dos padrões, dimensão da imagem (Opcional) e a opção de saída de dado.

\textbf{Output :} Os símbolos adquiridos ou o número de padrões formados.\\

\begin{lstlisting}
formationPatternsImage <-function(myImg,dimX,dimY,delX,delY,dimension=0,option=0){
  if(!dimension){
    dimension=dim(myImg)[1]
  }
  n_symbols = previusX = 0
  d = dimX*dimY
  myArray = orderm = array(0,d)
  p_patterns = matrix(nrow=dimension*dimension,ncol=dimX*dimY)
  grayScale = transformMatrix(myImg,dimension)
  for(i in 1:dimension){
    if(i+(delY*(dimY-1)) <= dimension){
      for(j in 1:dimension){
        if(j+(delX*(dimX-1)) <= dimension){ 
          x = j
          y = i
          var = 0
          for(a in 1:dimY){
            if(a > 1){
              y = y + delY
            }
            for(b in 1:dimX){
              if(b > 1){
                x = x +delX
              }else{
                previusX = x
              }
              var = var + 1
              myArray[var] = grayScale[y,x]
            }
            x = previusX
          }
          n_symbols = n_symbols + 1
          p_patterns[n_symbols,] = order(myArray)
        }else{
          break
        }
      }
    }else{
      break
    }
  }
  p_patterns=p_patterns[1:n_symbols,]
  if(!option){
    return(p_patterns)
  }else{
    return(n_symbols)
  }
}
\end{lstlisting}

\subsection{Função \textit{distributionImage}}

Sendo uma modificação da função \textit{distribution}, que retorna a distribuição de probabilidade e a frequência absoluta de uma série, a \textit{distributionImage} realiza os mesmos objetivos, entretanto possuindo como entrada dados bidimensionais e utilizando a função \textit{formationPatternsImage} para fornecer os padrões obtidos por estes.\\

\textbf{Input :} Imagem, dimensões X e y dos padrões, delays X e Y dos padrões, dimensão da imagem (Opcional) e a opção de saída de dado.

\textbf{Output :} Frequência absoluta ou distribuição de probabilidade.\\

 \begin{lstlisting}
  distributionImage<-function(myImg,dimX,dimY,delX,delY,dimension=0,option=1){  
    if(!dimension){
      dimension=dim(myImg)[1]
    }
    d = dimX*dimY
    fat = factorial(d)
    f_absolute = probability = rep(0,fat)  
    p_patterns <- formationPatternsImage(myImg,dimX,dimY,delX,delY,dimension)
    n_symbols <- formationPatternsImage(myImg,dimX,dimY,delX,delY,dimension, 1)
    symbols <- definePatterns(d)
    aux = rep(0,n_symbols)  
    for(i in 1:fat){
      for(j in 1:n_symbols){
        if(aux[j] == 0){
          if(all(p_patterns[j,] == symbols[i,])){ 
            f_absolute[i]=f_absolute[i]+1
            aux[j]=1
          }
        }
      }
    }
    probability = f_absolute/n_symbols
    if(!option){
      return(f_absolute)
    }else{
      return(probability)
    }
  }
\end{lstlisting}

\section{Conclusões\label{sec:conclusions}}

Assim, por meio da implementação das novas funcionalidades será possível realizar a formação dos padrões de dados bidimensionais e extrair informações destes, como os seus descritores causais. Entretanto algumas dúvidas foram adquiridas ao longo do período de implementação, dificultando a implementação de certas funcionalidades, como podemos ver a seguir:

\begin{itemize}
\item Por não saber o modo correto de demonstrar gráficamente dados bidimensionais não foi possível implementar uma função que pudesse fornecer o gráfico de tais dados;

\item Por não possui dados suficientes também não foi possível realizar a implementação da função do plano complexidade entropia de dados bidimensionais, uma vez que possuo apenas tais valores para dados unidimensionais.
\end{itemize}
		
\end{document}
