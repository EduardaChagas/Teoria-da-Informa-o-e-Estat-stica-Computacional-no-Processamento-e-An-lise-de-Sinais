\documentclass[12pt,letterpaper]{article}
\usepackage[utf8]{inputenc}
\usepackage[spanish, es-tabla]{babel}
\usepackage[version=3]{mhchem}
\usepackage[journal=jacs]{chemstyle}
\usepackage{amsmath}
\usepackage{amsfonts}
\usepackage{amssymb}
\usepackage{makeidx}
\usepackage{xcolor}
\usepackage[stable]{footmisc}
\usepackage[section]{placeins}
\usepackage{listings}
\usepackage{siunitx}
\usepackage{tabularx}
\sisetup{mode=text, output-decimal-marker = {,}, per-mode = symbol, qualifier-mode = phrase, qualifier-phrase = { de }, list-units = brackets, range-units = brackets, range-phrase = --}
\DeclareSIUnit[number-unit-product = \;] \atmosphere{atm}
\DeclareSIUnit[number-unit-product = \;] \pound{lb}
\DeclareSIUnit[number-unit-product = \;] \inch{"}
\DeclareSIUnit[number-unit-product = \;] \foot{ft}
\DeclareSIUnit[number-unit-product = \;] \yard{yd}
\DeclareSIUnit[number-unit-product = \;] \mile{mi}
\DeclareSIUnit[number-unit-product = \;] \pint{pt}
\DeclareSIUnit[number-unit-product = \;] \quart{qt}
\DeclareSIUnit[number-unit-product = \;] \flounce{fl-oz}
\DeclareSIUnit[number-unit-product = \;] \ounce{oz}
\DeclareSIUnit[number-unit-product = \;] \degreeFahrenheit{\SIUnitSymbolDegree F}
\DeclareSIUnit[number-unit-product = \;] \degreeRankine{\SIUnitSymbolDegree R}
\DeclareSIUnit[number-unit-product = \;] \usgallon{galón}
\DeclareSIUnit[number-unit-product = \;] \uma{uma}
\DeclareSIUnit[number-unit-product = \;] \ppm{ppm}
\DeclareSIUnit[number-unit-product = \;] \eqg{eq-g}
\DeclareSIUnit[number-unit-product = \;] \normal{\eqg\per\liter\of{solución}}
\DeclareSIUnit[number-unit-product = \;] \molal{\mole\per\kilo\gram\of{solvente}}
\usepackage{cancel}
\usepackage{graphicx}
\usepackage{lmodern}
\usepackage{fancyhdr}
\usepackage[left=4cm,right=2cm,top=3cm,bottom=3cm]{geometry}
\usepackage[backend=bibtex,style=chem-acs,biblabel=dot]{biblatex}
\addbibresource{references.bib}
\usepackage{titlesec}
\usepackage{enumitem}
\titleformat*{\section}{\bfseries\large}
\titleformat*{\subsection}{\bfseries\normalsize}
\usepackage{float}
\floatstyle{plaintop}
\newfloat{anexo}{thp}{anx}
\floatname{anexo}{Anexo}
\restylefloat{anexo}
\restylefloat{figure}
\usepackage[margin=10pt,labelfont=bf]{caption}
\usepackage{todonotes}
\usepackage[colorlinks=true, 
            linkcolor = blue,
            urlcolor  = blue,
            citecolor = black,
            anchorcolor = blue]{hyperref}
            
            
\begin{document}
\renewcommand{\labelitemi}{$\checkmark$}

\renewcommand{\CancelColor}{\color{red}}

\newcolumntype{L}[1]{>{\raggedright\let\newline\\\arraybackslash}m{#1}}

\newcolumntype{C}[1]{>{\centering\let\newline\\\arraybackslash}m{#1}}

\newcolumntype{R}[1]{>{\raggedleft\let\newline\\\arraybackslash}m{#1}}

\begin{center}
	\textbf{\LARGE{Time Series Analysis - Manual de utilização das funções desenvolvidas}}\\
	\vspace{7mm}
	\textbf{LaCCAN - Laboratório de computação científica e análise numérica}\\ 
	\vspace{4mm}
	\textbf{Eduarda Chagas, Alejandro Frery}\\
\end{center}

\vspace{7mm}

\section{Pacotes necessários}

Para o pleno funcionamento das funções desenvolvidas, será necessário que os seguintes pacotes estejam instalados no ambiente \textit{RStudio}:

\begin{itemize}
\item combinat
\item ggplot2
\item dygraphs
\item ggthemes
\end{itemize}


\section{Funções desenvolvidas}

%------------------------------------------------------------------------------------------------------------------

\hrulefill   

\begin{table}[!h]
\begin{center}
\begin{tabularx}{\textwidth}{ X X}
\hspace{0.5cm} Readtxt & \textit{Leitura de dados de um arquivo .txt}\\
\end{tabularx}
\end{center}
\end{table} 

\vspace{-0.5cm}

\hrulefill  

\vspace{0.5cm}

\textbf{Uso}

\begin{lstlisting}
   Readtxt(column)
\end{lstlisting}

\vspace{0.5cm}

\textbf{Argumentos}

\begin{table}[!h]
\begin{center}
\begin{tabularx}{\textwidth}{X X}
\hspace{0.5cm} \textit{column} & Coluna de dados que deseja ser lida pela função.\\
\end{tabularx}
\end{center}
\end{table} 

\newpage


%------------------------------------------------------------------------------------------------------------------

\hrulefill   

\begin{table}[!h]
\begin{center}
\begin{tabularx}{\textwidth}{ X X}
\hspace{0.5cm} Readcsv & \textit{Leitura de dados de um arquivo .csv}\\
\end{tabularx}
\end{center}
\end{table} 

\vspace{-0.5cm}

\hrulefill  

\vspace{0.5cm}

\textbf{Uso}

\begin{lstlisting}
   Readcsv(column,separator)
\end{lstlisting}

\vspace{0.5cm}

\textbf{Argumentos}

\begin{table}[!h]
\begin{center}
\begin{tabularx}{\textwidth}{X X}
\hspace{0.5cm} \textit{column} & Coluna de dados que deseja ser lida pela função.\\
\hspace{0.5cm} \textit{separator} & (Opcional) Caracter separador de campo. Os valores em cada linha do arquivo são separados por esse caracter. É colocado por default o ";".\\
\end{tabularx}
\end{center}
\end{table} 


%------------------------------------------------------------------------------------------------------------------

\hrulefill   

\begin{table}[!h]
\begin{center}
\begin{tabularx}{\textwidth}{ X X}
\hspace{0.5cm} equalitiesValues & \textit{Percentual de dados repetidos em uma dada série temporal}\\
\end{tabularx}
\end{center}
\end{table} 

\vspace{-0.5cm}

\hrulefill  

\vspace{0.5cm}

\textbf{Uso}

\begin{lstlisting}
   equalitiesValues(serie)
\end{lstlisting}

\vspace{0.5cm}

\textbf{Argumentos}

\begin{table}[!h]
\begin{center}
\begin{tabularx}{\textwidth}{X X}
\hspace{0.5cm} \textit{serie} & Um vetor numérico onde estará instânciada a série temporal que deve ser avaliada pela função.\\
\end{tabularx}
\end{center}
\end{table} 

\newpage

%------------------------------------------------------------------------------------------------------------------

\hrulefill   

\begin{table}[!h]
\begin{center}
\begin{tabularx}{\textwidth}{X X}
\hspace{0.5cm} removeDuplicate & \textit{Remove os dados duplicados de uma série temporal}\\
\end{tabularx}
\end{center}
\end{table} 

\vspace{-0.5cm}

\hrulefill  

\vspace{0.5cm}

\textbf{Uso}

\begin{lstlisting}
   removeDuplicate(serie)
\end{lstlisting}

\vspace{0.5cm}

\textbf{Argumentos}

\begin{table}[!h]
\begin{center}
\begin{tabularx}{\textwidth}{X X}
\hspace{0.5cm} \textit{serie} & Um vetor numérico onde estará instânciada a série temporal que deve ser avaliada pela função.\\
\end{tabularx}
\end{center}
\end{table} 

%------------------------------------------------------------------------------------------------------------------

\hrulefill   

\begin{table}[!h]
\begin{center}
\begin{tabularx}{\textwidth}{X X}
\hspace{0.5cm} distribution & \textit{Distribuição de probabilidade de Bandt and Pompe}\\
\end{tabularx}
\end{center}
\end{table} 

\vspace{-0.5cm}

\hrulefill  

\vspace{0.5cm}

\textbf{Uso}

\begin{lstlisting}
   distribution(serie,dimension,delay)
\end{lstlisting}

\vspace{0.5cm}

\textbf{Argumentos}

\begin{table}[!h]
\begin{center}
\begin{tabularx}{\textwidth}{X X}
\hspace{0.5cm} \textit{serie} \vspace{0.5cm}& Um vetor numérico onde estará instânciada a série temporal que deve ser avaliada pela função.\vspace{0.5cm}\\
\hspace{0.5cm} \textit{dimension} \vspace{0.5cm}& Dimensão dos padrões ordinais.\vspace{0.5cm}\\
\hspace{0.5cm} \textit{delay} & Delay utilizado na formação dos padrões.\\
\end{tabularx}
\end{center}
\end{table} 

\newpage

%------------------------------------------------------------------------------------------------------------------

\hrulefill   

\begin{table}[!h]
\begin{center}
\begin{tabularx}{\textwidth}{X X}
\hspace{0.5cm} WPE & Distribuição de probabilidade de Bandt and Pompe Weigth.\\
\end{tabularx}
\end{center}
\end{table} 

\vspace{-0.5cm}

\hrulefill  

\vspace{0.5cm}

\textbf{Uso}

\begin{lstlisting}
   WPE(serie,dimension,delay)
\end{lstlisting}

\vspace{0.5cm}


\textbf{Argumentos}

\begin{table}[!h]
\begin{center}
\begin{tabularx}{\textwidth}{X X}
\hspace{0.5cm} \textit{serie} \vspace{0.5cm}& Um vetor numérico onde estará instânciada a série temporal que deve ser avaliada pela função.\vspace{0.5cm}\\
\hspace{0.5cm} \textit{dimension} \vspace{0.5cm}& Dimensão dos padrões ordinais.\vspace{0.5cm}\\
\hspace{0.5cm} \textit{delay} & Delay utilizado na formação dos padrões.\\
\end{tabularx}
\end{center}
\end{table} 

%---------------------------------------------------------------------------------------------------------------

\hrulefill   

\begin{table}[!h]
\begin{center}
\begin{tabularx}{\textwidth}{ X X}
  \hspace{0.5cm} shannonEntropy & \textit{Entropia de Shannon de uma dada distribuição de probabilidade}\\
\end{tabularx}
\end{center}
\end{table} 

\vspace{-0.5cm}

\hrulefill  

\vspace{0.5cm}

\textbf{Uso}

\begin{lstlisting}
   shannonEntropy(p)
\end{lstlisting}

\vspace{0.5cm}

\textbf{Argumentos}

\begin{table}[!h]
\begin{center}
\begin{tabularx}{\textwidth}{X X}
\hspace{0.5cm} \textit{p} & Uma distribuição de padrão ordinal.\\
\end{tabularx}
\end{center}
\end{table} 

\newpage
%---------------------------------------------------------------------------------------------------------------

\hrulefill   

\begin{table}[!h]
\begin{center}
\begin{tabularx}{\textwidth}{ X X}
\hspace{0.5cm} shannonNormalized & \textit{Entropia normalizada de Shannon de uma distribuição de probabilidade}\\
\end{tabularx}
\end{center}
\end{table} 

\vspace{-0.5cm}

\hrulefill  

\vspace{0.5cm}

\textbf{Uso}

\begin{lstlisting}
   shannonNormalized(p)
\end{lstlisting}

\vspace{0.5cm}

\textbf{Argumentos}

\begin{table}[!h]
\begin{center}
\begin{tabularx}{\textwidth}{X X}
\hspace{0.5cm} \textit{p} & Uma distribuição de padrão ordinal.\\
\end{tabularx}
\end{center}
\end{table} 

%---------------------------------------------------------------------------------------------------------------
\hrulefill   

\begin{table}[!h]
\begin{center}
\begin{tabularx}{\textwidth}{ X X}
\hspace{0.5cm} tsallisEntropy & \textit{Entropia de Tsallis de uma dada distribuição de probabilidade}\\
\end{tabularx}
\end{center}
\end{table} 

\vspace{-0.5cm}

\hrulefill  

\vspace{0.5cm}

\textbf{Uso}

\begin{lstlisting}
   tsallisEntropy(p,q)
\end{lstlisting}

\vspace{0.5cm}

\textbf{Argumentos}

\begin{table}[!h]
\begin{center}
\begin{tabularx}{\textwidth}{X X}
\hspace{0.5cm} \textit{p} \vspace{0.5cm}& Uma distribuição de padrão ordinal.\vspace{0.5cm}\\
\hspace{0.5cm} \textit{q} \vspace{0.5cm}& A ordem da entropia. Permite apenas números positivos.\vspace{0.5cm}\\
\end{tabularx}
\end{center}
\end{table} 

\newpage
%---------------------------------------------------------------------------------------------------------------

\hrulefill   

\begin{table}[!h]
\begin{center}
\begin{tabularx}{\textwidth}{ X X}
\hspace{0.5cm} tsallisNormalized & \textit{Entropia normalizada de Tsallis de uma distribuição de probabilidade}\\
\end{tabularx}
\end{center}
\end{table} 

\vspace{-0.5cm}

\hrulefill  

\vspace{0.5cm}

\textbf{Uso}

\begin{lstlisting}
   tsallisNormalized(p,q)
\end{lstlisting}

\vspace{0.5cm}

\textbf{Argumentos}

\begin{table}[!h]
\begin{center}
\begin{tabularx}{\textwidth}{X X}
\hspace{0.5cm} \textit{p} \vspace{0.5cm}& Uma distribuição de padrão ordinal.\vspace{0.5cm}\\
\hspace{0.5cm} \textit{q} \vspace{0.5cm}&  A ordem da entropia. Permite apenas números positivos.\vspace{0.5cm}\\
\end{tabularx}
\end{center}
\end{table} 

%---------------------------------------------------------------------------------------------------------------

\hrulefill   

\begin{table}[!h]
\begin{center}
\begin{tabularx}{\textwidth}{ X X}
\hspace{0.5cm} renyiEntropy & \textit{Entropia de Renyi de uma distribuição de probabilidade}\\
\end{tabularx}
\end{center}
\end{table} 

\vspace{-0.5cm}

\hrulefill  

\vspace{0.5cm}

\textbf{Uso}

\begin{lstlisting}
   renyiEntropy(p,q)
\end{lstlisting}

\vspace{0.5cm}

\textbf{Argumentos}

\begin{table}[!h]
\begin{center}
\begin{tabularx}{\textwidth}{X X}
\hspace{0.5cm} \textit{p} \vspace{0.5cm}& Uma distribuição de padrão ordinal.\vspace{0.5cm}\\
\hspace{0.5cm} \textit{q} \vspace{0.5cm}& A ordem da entropia. Permite apenas números positivos.\vspace{0.5cm}\\
\end{tabularx}
\end{center}
\end{table} 
\newpage

%---------------------------------------------------------------------------------------------------------------

\hrulefill   

\begin{table}[!h]
\begin{center}
\begin{tabularx}{\textwidth}{ X X}
\hspace{0.5cm} renyiNormalized & \textit{Entropia normalizada de Renyi de uma distribuição de probabilidade}\\
\end{tabularx}
\end{center}
\end{table} 

\vspace{-0.5cm}

\hrulefill  

\vspace{0.5cm}

\textbf{Uso}

\begin{lstlisting}
   renyiNormalized(p,q)
\end{lstlisting}

\vspace{0.5cm}

\textbf{Argumentos}

\begin{table}[!h]
\begin{center}
\begin{tabularx}{\textwidth}{X X}
\hspace{0.5cm} \textit{p} \vspace{0.5cm}& Uma distribuição de padrão ordinal.\vspace{0.5cm}\\
\hspace{0.5cm} \textit{q} \vspace{0.5cm}& A ordem da entropia. Permite apenas números positivos.\vspace{0.5cm}\\
\end{tabularx}
\end{center}
\end{table} 

%---------------------------------------------------------------------------------------------------------------

\hrulefill   

\begin{table}[!h]
\begin{center}
\begin{tabularx}{\textwidth}{ X X}
\hspace{0.5cm} PME & \textit{Entropia entropia de mínima permutação de uma distribuição de probabilidade}\\
\end{tabularx}
\end{center}
\end{table} 

\vspace{-0.5cm}

\hrulefill  

\vspace{0.5cm}

\textbf{Uso}

\begin{lstlisting}
   PME(p)
\end{lstlisting}

\vspace{0.5cm}

\textbf{Argumentos}

\begin{table}[!h]
\begin{center}
\begin{tabularx}{\textwidth}{X X}
\hspace{0.5cm} \textit{p} & Uma distribuição de padrão ordinal.\\
\end{tabularx}
\end{center}
\end{table} 
\newpage
%------------------------------------------------------------------------------------------------------

\hrulefill   

\begin{table}[!h]
\begin{center}
\begin{tabularx}{\textwidth}{ X X}
\hspace{0.5cm} euclidianDistance & \textit{Distância euclidiana de uma dada distribuição de probabilidade e a distribuição de probabilidade uniforme}\\
\end{tabularx}
\end{center}
\end{table} 

\vspace{-0.5cm}

\hrulefill  

\vspace{0.5cm}

\textbf{Uso}

\begin{lstlisting}
   euclidianDistance(p)
\end{lstlisting}

\vspace{0.5cm}

\textbf{Argumentos}

\begin{table}[!h]
\begin{center}
\begin{tabularx}{\textwidth}{X X}
\hspace{0.5cm} \textit{p} & Uma distribuição de padrão ordinal.\\
\end{tabularx}
\end{center}
\end{table} 

%---------------------------------------------------------------------------------------------------------------

\hrulefill   

\begin{table}[!h]
\begin{center}
\begin{tabularx}{\textwidth}{ X X}
\hspace{0.5cm} squaredDistance & \textit{Distância euclidiana quadrada de uma dada distribuição de probabilidade e a distribuição de probabilidade uniforme}\\
\end{tabularx}
\end{center}
\end{table} 

\vspace{-0.5cm}

\hrulefill  

\vspace{0.5cm}

\textbf{Uso}

\begin{lstlisting}
   squaredDistance(p)
\end{lstlisting}

\vspace{0.5cm}

\textbf{Argumentos}

\begin{table}[!h]
\begin{center}
\begin{tabularx}{\textwidth}{X X}
\hspace{0.5cm} \textit{p} & Uma distribuição de padrão ordinal.\\
\end{tabularx}
\end{center}
\end{table} 

\newpage

%---------------------------------------------------------------------------------------------------------------

\hrulefill   

\begin{table}[!h]
\begin{center}
\begin{tabularx}{\textwidth}{ X X}
\hspace{0.5cm} manhattanDistance & \textit{Distância de Manhattan de uma dada distribuição de probabilidade e a distribuição de probabilidade uniforme}\\
\end{tabularx}
\end{center}
\end{table} 

\vspace{-0.5cm}

\hrulefill  

\vspace{0.5cm}

\textbf{Uso}

\begin{lstlisting}
   manhattanDistance(p)
\end{lstlisting}

\vspace{0.5cm}

\textbf{Argumentos}

\begin{table}[!h]
\begin{center}
\begin{tabularx}{\textwidth}{X X}
\hspace{0.5cm} \textit{p} & Uma distribuição de padrão ordinal.\\
\end{tabularx}
\end{center}
\end{table} 
%---------------------------------------------------------------------------------------------------------------

\hrulefill   

\begin{table}[!h]
\begin{center}
\begin{tabularx}{\textwidth}{ X X}
\hspace{0.5cm} chebyshevDistance & \textit{Distância de Chebyshev de uma dada distribuição de probabilidade e a distribuição de probabilidade uniforme}\\
\end{tabularx}
\end{center}
\end{table} 

\vspace{-0.5cm}

\hrulefill  

\vspace{0.5cm}

\textbf{Uso}

\begin{lstlisting}
   chebyshevDistance(p)
\end{lstlisting}

\vspace{0.5cm}

\textbf{Argumentos}

\begin{table}[!h]
\begin{center}
\begin{tabularx}{\textwidth}{X X}
\hspace{0.5cm} \textit{p} & Uma distribuição de padrão ordinal.\\
\end{tabularx}
\end{center}
\end{table} 

\newpage

%---------------------------------------------------------------------------------------------------------------

\hrulefill   

\begin{table}[!h]
\begin{center}
\begin{tabularx}{\textwidth}{ X X}
\hspace{0.5cm} hellingerDistance & \textit{Distância  de Hellinger de uma dada distribuição de probabilidade e a distribuição de probabilidade uniforme}\\
\end{tabularx}
\end{center}
\end{table} 

\vspace{-0.5cm}

\hrulefill  

\vspace{0.5cm}

\textbf{Uso}

\begin{lstlisting}
   hellingerDistance(p)
\end{lstlisting}

\vspace{0.5cm}

\textbf{Argumentos}

\begin{table}[!h]
\begin{center}
\begin{tabularx}{\textwidth}{X X}
\hspace{0.5cm} \textit{p} & Uma distribuição de padrão ordinal.\\
\end{tabularx}
\end{center}
\end{table} 

%---------------------------------------------------------------------------------------------------------------

\hrulefill   

\begin{table}[!h]
\begin{center}
\begin{tabularx}{\textwidth}{ X X}
\hspace{0.5cm} jensenDivergence & \textit{Medida de desequilíbrio generalizado para distribuições de probabilidade com base na divergência de Jensen-Shannon}\\
\end{tabularx}
\end{center}
\end{table} 

\vspace{-0.5cm}

\hrulefill  

\vspace{0.5cm}

\textbf{Uso}

\begin{lstlisting}
   jensenDivergence(p)
\end{lstlisting}

\vspace{0.5cm}

\textbf{Argumentos}

\begin{table}[!h]
\begin{center}
\begin{tabularx}{\textwidth}{X X}
\hspace{0.5cm} \textit{p} & Uma distribuição de padrão ordinal.\\
\end{tabularx}
\end{center}
\end{table} 

\newpage
%---------------------------------------------------------------------------------------------------------------

\hrulefill   

\begin{table}[!h]
\begin{center}
\begin{tabularx}{\textwidth}{ X X}
\hspace{0.5cm} woottersDistance & \textit{Distância de Wootters de uma dada distribuição de probabilidade e a distribuição de probabilidade uniforme}\\
\end{tabularx}
\end{center}
\end{table} 

\vspace{-0.5cm}

\hrulefill  

\vspace{0.5cm}

\textbf{Uso}

\begin{lstlisting}
   woottersDistance(p)
\end{lstlisting}

\vspace{0.5cm}

\textbf{Argumentos}

\begin{table}[!h]
\begin{center}
\begin{tabularx}{\textwidth}{X X}
\hspace{0.5cm} \textit{p} & Uma distribuição de padrão ordinal.\\
\end{tabularx}
\end{center}
\end{table} 
%---------------------------------------------------------------------------------------------------------------

\hrulefill   

\begin{table}[!h]
\begin{center}
\begin{tabularx}{\textwidth}{ X X}
\hspace{0.5cm} kullbackDivergence & \textit{Medida de desequilíbrio generalizado para distribuições de probabilidade com base na divergência de kullback-Leibler}\\
\end{tabularx}
\end{center}
\end{table} 

\vspace{-0.5cm}

\hrulefill  

\vspace{0.5cm}

\textbf{Uso}

\begin{lstlisting}
   kullbackDivergence(p)
\end{lstlisting}

\vspace{0.5cm}

\textbf{Argumentos}

\begin{table}[!h]
\begin{center}
\begin{tabularx}{\textwidth}{X X}
\hspace{0.5cm} \textit{p} & Uma distribuição de padrão ordinal.\\
\end{tabularx}
\end{center}
\end{table} 

\newpage

%---------------------------------------------------------------------------------------------------------------

\hrulefill   

\begin{table}[!h]
\begin{center}
\begin{tabularx}{\textwidth}{ X X}
\hspace{0.5cm} bhattacharyyaDistance & \textit{Distância de Bhattacharyya de uma dada distribuição de probabilidade e a distribuição de probabilidade uniforme}\\
\end{tabularx}
\end{center}
\end{table} 

\vspace{-0.5cm}

\hrulefill  

\vspace{0.5cm}

\textbf{Uso}

\begin{lstlisting}
   bhattacharyyaDistance(p)
\end{lstlisting}

\vspace{0.5cm}

\textbf{Argumentos}

\begin{table}[!h]
\begin{center}
\begin{tabularx}{\textwidth}{X X}
\hspace{0.5cm} \textit{p} & Uma distribuição de padrão ordinal.\\
\end{tabularx}
\end{center}
\end{table} 
%----------------------------------------------------------------------------------------------------------

\hrulefill   

\begin{table}[!h]
\begin{center}
\begin{tabularx}{\textwidth}{ X X}
\hspace{0.5cm} Ccomplexity & \textit{Complexidade estatística de uma distribuição de probabilidade}\\
\end{tabularx}
\end{center}
\end{table} 

\vspace{-0.5cm}

\hrulefill  

\vspace{0.5cm}

\textbf{Uso}

\begin{lstlisting}
   Ccomplexity(p)
\end{lstlisting}

\vspace{0.5cm}

\textbf{Argumentos}

\begin{table}[!h]
\begin{center}
\begin{tabularx}{\textwidth}{X X}
\hspace{0.5cm} \textit{p} & Uma distribuição de padrão ordinal.\\
\end{tabularx}
\end{center}
\end{table} 

\newpage
%----------------------------------------------------------------------------------------------------------

\hrulefill   

\begin{table}[!h]
\begin{center}
\begin{tabularx}{\textwidth}{ X X}
\hspace{0.5cm} timeSeries & \textit{Gráfico da série temporal}\\
\end{tabularx}
\end{center}
\end{table} 

\vspace{-0.5cm}

\hrulefill  

\vspace{0.5cm}

\textbf{Uso}

\begin{lstlisting}
   timeSeries(serie)
\end{lstlisting}

\vspace{0.5cm}

\textbf{Argumentos}

\begin{table}[!h]
\begin{center}
\begin{tabularx}{\textwidth}{X X}
\hspace{0.5cm} \textit{serie} & Um vetor numérico onde estará instânciada a série temporal que deve ser avaliada pela função.\\
\end{tabularx}
\end{center}
\end{table} 
%---------------------------------------------------------------------------------------------------------------

\hrulefill   

\begin{table}[!h]
\begin{center}
\begin{tabularx}{\textwidth}{ X X}
\hspace{0.5cm} histogram & \textit{Histograma dos padrões de Bandt and Pompe de uma série temporal}\\
\end{tabularx}
\end{center}
\end{table} 

\vspace{-0.5cm}

\hrulefill  

\vspace{0.5cm}

\textbf{Uso}

\begin{lstlisting}
   histogram(serie,dimension,delay)
\end{lstlisting}

\vspace{0.5cm}

\textbf{Argumentos}

\begin{table}[!h]
\begin{center}
\begin{tabularx}{\textwidth}{X X}
\hspace{0.5cm} \textit{serie} \vspace{0.5cm}& Um vetor numérico onde estará instânciada a série temporal que deve ser avaliada pela função.\vspace{0.5cm}\\
\hspace{0.5cm} \textit{dimension} \vspace{0.5cm}& Dimensão dos padrões ordinais.\vspace{0.5cm}\\
\hspace{0.5cm} \textit{delay} \vspace{0.5cm}& Delay utilizado na formação dos padrões.\vspace{0.5cm}\\
\end{tabularx}
\end{center}
\end{table} 

\newpage
%---------------------------------------------------------------------------------------------------------------

\hrulefill   

\begin{table}[!h]
\begin{center}
\begin{tabularx}{\textwidth}{ X X}
\hspace{0.5cm} patternsOnGraph & \textit{Localiza os pontos de uma série temporal pertencentes a um certo padrão ordinal}
\end{tabularx}
\end{center}
\end{table} 

\vspace{-0.5cm}

\hrulefill  

\vspace{0.5cm}

\textbf{Uso}

\begin{lstlisting}
   patternsOnGraph(serie,dimension,delay,pattern)
\end{lstlisting}

\vspace{0.5cm}

\textbf{Argumentos}

\begin{table}[!h]
\begin{center}
\begin{tabularx}{\textwidth}{X X}
\hspace{0.5cm} \textit{serie} \vspace{0.5cm}& Um vetor numérico onde estará instânciada a série temporal que deve ser avaliada pela função.\vspace{0.5cm}\\
\hspace{0.5cm} \textit{dimension} \vspace{0.5cm}& Dimensão dos padrões ordinais.\vspace{0.5cm}\\
\hspace{0.5cm} \textit{delay} \vspace{0.5cm}& Delay utilizado na formação dos padrões.\vspace{0.5cm}\\
\hspace{0.5cm} \textit{pattern} \vspace{0.5cm}& Padrão que deverá ser analisado. Valor baseado nos padrões demonstrados no histograma.\vspace{0.5cm}\\
\end{tabularx}
\end{center}
\end{table} 
\newpage
%---------------------------------------------------------------------------------------------------------------

\hrulefill   

\begin{table}[!h]
\begin{center}
\begin{tabularx}{\textwidth}{ X X}
\hspace{0.5cm} entropyPlane & \textit{Plota o gráfico da entropia de uma certa série temporal ou seus particionamentos}\\
\end{tabularx}
\end{center}
\end{table} 

\vspace{-0.5cm}

\hrulefill  

\vspace{0.5cm}

\textbf{Uso}

\begin{lstlisting}
   entropyPlane(serie,partitions,dimension,delay,
   			distribution,option,q)
\end{lstlisting}

\vspace{0.5cm}

\textbf{Argumentos}

\begin{table}[!h]
\begin{center}
\begin{tabularx}{\textwidth}{X X}
\hspace{0.5cm} \textit{serie} \vspace{0.5cm}& Um vetor numérico onde estará instânciada a série temporal que deve ser avaliada pela função.\vspace{0.5cm}\\
\hspace{0.5cm} \textit{partitions} \vspace{0.5cm}& Quantidade de partições que a série deve ser dividida para a análise.\vspace{0.5cm}\\
\hspace{0.5cm} \textit{dimension} \vspace{0.5cm}& Dimensão dos padrões ordinais.\vspace{0.5cm}\\
\hspace{0.5cm} \textit{delay} \vspace{0.5cm}& Delay utilizado na formação dos padrões.\vspace{0.5cm}\\
\hspace{0.5cm} \textit{distribution} \vspace{0.5cm}& Distribuição que deve ser utilizada. O parâmetro deverá ser 1 para a distribuição de Bandt and Pompe, caso contrário a distribuição que será aplicada será a de Bandt and Pompe weigth.\vspace{0.5cm}\\
\hspace{0.5cm} \textit{option} \vspace{0.5cm}& Entropia que deve ser analisada. O parâmetro deve ser 1 para a entropia de shannon, 2 para a entropia de Tsallis ou 3 para a entropia de Renyi. Caso contrário, deverá ser aplicada a min entropy.\vspace{0.5cm}\\
\hspace{0.5cm} \textit{q} \vspace{0.5cm}& (Pode não ser necessário dependendo da entropia selecionada) Ordem da entropia.\vspace{0.5cm}\\
\end{tabularx}
\end{center}
\end{table} 

\newpage
%---------------------------------------------------------------------------------------------------------------

\hrulefill   

\begin{table}[!h]
\begin{center}
\begin{tabularx}{\textwidth}{ X X}
\hspace{0.5cm} distancePlane & \textit{Plota o gráfico da distância estocástica de uma certa série temporal ou seus particionamentos}\\
\end{tabularx}
\end{center}
\end{table} 

\vspace{-0.5cm}

\hrulefill  

\vspace{0.5cm}

\textbf{Uso}

\begin{lstlisting}
   distancePlane<-function(serie,partition,dimension,delay,
   				optionD=1,optionP=1,q=1){
\end{lstlisting}

\vspace{0.5cm}

\textbf{Argumentos}

\begin{table}[!h]
\begin{center}
\begin{tabularx}{\textwidth}{X X}
\hspace{0.5cm} \textit{serie} \vspace{0.5cm}& Um vetor numérico onde estará instânciada a série temporal que deve ser avaliada pela função.\vspace{0.5cm}\\
\hspace{0.5cm} \textit{partition} \vspace{0.5cm}& Quantidade de partições que a série deve ser dividida para a análise.\vspace{0.5cm}\\
\hspace{0.5cm} \textit{dimension} \vspace{0.5cm}& Dimensão dos padrões ordinais.\vspace{0.5cm}\\
\hspace{0.5cm} \textit{delay} \vspace{0.5cm}& Delay utilizado na formação dos padrões.\vspace{0.5cm}\\
\hspace{0.5cm} \textit{optionD} \vspace{0.5cm}& Distância estocástica que deve ser analisada. Devem ser usados os seguintes valores para acessar as opções disponíveis: Euclidiana (1), euclidiana quadrática (2), manhattan (3), chebyshev (4), divergência de Kullback-Leibler (5), helinger (6), divergência de Jensen Shannon (7), wootters (8) e bhattacharyya (9)\vspace{0.5cm}\\
\hspace{0.5cm} \textit{optionP} \vspace{0.5cm}& Distribuição que deve ser utilizada. O parâmetro deverá ser 1 para a distribuição de Bandt and Pompe, caso contrário a distribuição que será aplicada será a de Bandt and Pompe weigth.\vspace{0.5cm}\\
\end{tabularx}
\end{center}
\end{table} 

\begin{table}[!h]
\begin{center}
\begin{tabularx}{\textwidth}{X X}
\hspace{0.5cm} \textit{q} \vspace{0.5cm}& (Pode não ser necessário dependendo da entropia selecionada) Ordem da entropia.\vspace{0.5cm}\\
\end{tabularx}
\end{center}
\end{table} 

\newpage

%---------------------------------------------------------------------------------------------------------------

\hrulefill   

\begin{table}[!h]
\begin{center}
\begin{tabularx}{\textwidth}{ X X}
\hspace{0.5cm} partitionMPR & \textit{Gráfico entropia complexidade de uma série temporal ou de seus particionamentos}\\
\end{tabularx}
\end{center}
\end{table} 

\vspace{-0.5cm}

\hrulefill  

\vspace{0.5cm}

\textbf{Uso}

\begin{lstlisting}
   partitionMPR(serie,dimension,delay,partitions)
\end{lstlisting}

\vspace{0.5cm}

\textbf{Argumentos}

\begin{table}[!h]
\begin{center}
\begin{tabularx}{\textwidth}{X X}
\hspace{0.5cm} \textit{serie} \vspace{0.5cm}& Um vetor numérico onde estará instânciada a série temporal que deve ser avaliada pela função.\vspace{0.5cm}\\
\hspace{0.5cm} \textit{dimension} \vspace{0.5cm}& Dimensão dos padrões ordinais.\vspace{0.5cm}\\
\hspace{0.5cm} \textit{delay} \vspace{0.5cm}& Delay utilizado na formação dos padrões.\vspace{0.5cm}\\
\hspace{0.5cm} \textit{partitions} \vspace{0.5cm}& Quantidade de partições que a série deve ser dividida para a análise.\vspace{0.5cm}\\
\end{tabularx}
\end{center}
\end{table} 

%-----------------------------------------------------------------------------------------------

\hrulefill   

\begin{table}[!h]
\begin{center}
\begin{tabularx}{\textwidth}{ X X}
\hspace{0.5cm} saxPlot & \textit{Representação gráfica da classificação da série de acordo com o método Symbolic Aggregate Approximation}\\
\end{tabularx}
\end{center}
\end{table} 

\vspace{-0.5cm}

\hrulefill  

\vspace{0.5cm}

\textbf{Uso}

\begin{lstlisting}
   saxPlot(serie,letters,partitions)
\end{lstlisting}

\vspace{1.5cm}

\textbf{Argumentos}

\begin{table}[!h]
\begin{center}
\begin{tabularx}{\textwidth}{X X}
\hspace{0.5cm} \textit{serie} \vspace{0.5cm}& Um vetor numérico onde estará instânciada a série temporal que deve ser avaliada pela função.\vspace{0.5cm}\\
\hspace{0.5cm} \textit{letters} \vspace{0.5cm}&  O número de letras.\vspace{0.5cm}\\
\hspace{0.5cm} \textit{partitions} \vspace{0.5cm}& Quantidade de partições que a série deve ser dividida para a análise.\vspace{0.5cm}\\
\end{tabularx}
\end{center}
\end{table} 

%---------------------------------------------------------------------------------------------------------------

\hrulefill   

\begin{table}[!h]
\begin{center}
\begin{tabularx}{\textwidth}{ X X}
\hspace{0.5cm} PIP & \textit{Pontos encontrados na série fornecida pela técnica Perceptually Important Points}\\
\end{tabularx}
\end{center}
\end{table} 

\vspace{-0.5cm}

\hrulefill  

\vspace{0.5cm}
  
\textbf{Uso}

\begin{lstlisting}
   PIP(serie,numberOfPoints)
\end{lstlisting}

\vspace{0.5cm}

\textbf{Argumentos}

\begin{table}[!h]
\begin{center}
\begin{tabularx}{\textwidth}{X X}
\hspace{0.5cm} \textit{serie} \vspace{0.5cm}& Um vetor numérico onde estará instânciada a série temporal que deve ser avaliada pela função.\vspace{0.5cm}\\
\hspace{0.5cm} \textit{numberOfPoints} \vspace{0.5cm}& Números de pontos que devem ser encontrados pela função.\vspace{0.5cm}\\
\end{tabularx}
\end{center}
\end{table} 

\newpage
%---------------------------------------------------------------------------------------------------------------

\hrulefill   

\begin{table}[!h]
\begin{center}
\begin{tabularx}{\textwidth}{ X X}
\hspace{0.5cm} plotPAA & \textit{Exibe o gráfico dos valores adquiridos após o cálculo da Piecewise aggregate approximation}\\
\end{tabularx}
\end{center}
\end{table} 

\vspace{-0.5cm}

\hrulefill  

\vspace{0.5cm}
  
\textbf{Uso}

\begin{lstlisting}
   plotPAA(serie,partitions)
\end{lstlisting}

\vspace{0.5cm}

\textbf{Argumentos}

\begin{table}[!h]
\begin{center}
\begin{tabularx}{\textwidth}{X X}
\hspace{0.5cm} \textit{serie} \vspace{0.5cm}& Um vetor numérico onde estará instânciada a série temporal que deve ser avaliada pela função.\vspace{0.5cm}\\
\hspace{0.5cm} \textit{partitions} \vspace{0.5cm}& Quantidade de partições que a série deve ser dividida para a análise.\vspace{0.5cm}\\
\end{tabularx}
\end{center}
\end{table} 

\end{document}
