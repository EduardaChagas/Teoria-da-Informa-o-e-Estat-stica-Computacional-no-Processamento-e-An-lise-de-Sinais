\documentclass[12pt,letterpaper]{article}
\usepackage[utf8]{inputenc}
\usepackage[portuguese]{babel}
\usepackage[version=3]{mhchem}
\usepackage[journal=jacs]{chemstyle}
\usepackage{amsmath}
\usepackage{amsfonts}
\usepackage{amssymb}
\usepackage{makeidx}
\usepackage{xcolor}
\usepackage[stable]{footmisc}
\usepackage[section]{placeins}
\usepackage{listings}
\usepackage{siunitx}
\usepackage{tabularx}
\sisetup{mode=text, output-decimal-marker = {,}, per-mode = symbol, qualifier-mode = phrase, qualifier-phrase = { de }, list-units = brackets, range-units = brackets, range-phrase = --}
\DeclareSIUnit[number-unit-product = \;] \atmosphere{atm}
\DeclareSIUnit[number-unit-product = \;] \pound{lb}
\DeclareSIUnit[number-unit-product = \;] \inch{"}
\DeclareSIUnit[number-unit-product = \;] \foot{ft}
\DeclareSIUnit[number-unit-product = \;] \yard{yd}
\DeclareSIUnit[number-unit-product = \;] \mile{mi}
\DeclareSIUnit[number-unit-product = \;] \pint{pt}
\DeclareSIUnit[number-unit-product = \;] \quart{qt}
\DeclareSIUnit[number-unit-product = \;] \flounce{fl-oz}
\DeclareSIUnit[number-unit-product = \;] \ounce{oz}
\DeclareSIUnit[number-unit-product = \;] \degreeFahrenheit{\SIUnitSymbolDegree F}
\DeclareSIUnit[number-unit-product = \;] \degreeRankine{\SIUnitSymbolDegree R}
\DeclareSIUnit[number-unit-product = \;] \usgallon{galón}
\DeclareSIUnit[number-unit-product = \;] \uma{uma}
\DeclareSIUnit[number-unit-product = \;] \ppm{ppm}
\DeclareSIUnit[number-unit-product = \;] \eqg{eq-g}
\DeclareSIUnit[number-unit-product = \;] \normal{\eqg\per\liter\of{solución}}
\DeclareSIUnit[number-unit-product = \;] \molal{\mole\per\kilo\gram\of{solvente}}
\usepackage{cancel}
\usepackage{graphicx}
\usepackage{lmodern}
\usepackage{fancyhdr}
\usepackage[left=4cm,right=2cm,top=3cm,bottom=3cm]{geometry}
\usepackage{titlesec}
\usepackage{enumitem}
\titleformat*{\section}{\bfseries\large}
\titleformat*{\subsection}{\bfseries\normalsize}
\usepackage{float}
\floatstyle{plaintop}
\newfloat{anexo}{thp}{anx}
\floatname{anexo}{Anexo}
\restylefloat{anexo}
\restylefloat{figure}
\usepackage[margin=10pt,labelfont=bf]{caption}
\usepackage{todonotes}
\usepackage[colorlinks=true, 
            linkcolor = blue,
            urlcolor  = blue,
            citecolor = black,
            anchorcolor = blue]{hyperref}
            
            
\begin{document}
\renewcommand{\labelitemi}{$\checkmark$}

\renewcommand{\CancelColor}{\color{red}}

\newcolumntype{L}[1]{>{\raggedright\let\newline\\\arraybackslash}m{#1}}

\newcolumntype{C}[1]{>{\centering\let\newline\\\arraybackslash}m{#1}}

\newcolumntype{R}[1]{>{\raggedleft\let\newline\\\arraybackslash}m{#1}}

\begin{center}
	\textbf{\LARGE{Package ``DataDriven''}}\\
	\vspace{7mm}
	\textbf{Eduarda Chagas, Alejandro Frery}\\ 
	\vspace{4mm}
	\textbf{LaCCAN - UFAL}\\
\end{center}

\vspace{7mm}

\textbf{Version}: 0.0.0.9000\\

\textbf{Title}: Bandt and pompe symbolization dynamics for time series with tied values: a data-driven approach\\

\textbf{Description}: An implementation of a new data-driven imputation to treat time series with ties values present in the method introduced by Bandt and Pompe (Bandt and Pompe (2002) <DOI: 10.1103 / PhysRevLett.88.174102>).\\

\textbf{Imports}: combinat, gtools, stats\\

\textbf{License}: GPL-3\\

\textbf{Encoding}: UTF-8\\

\textbf{LazyData}: true\\

\textbf{RoxygenNote}: 6.0.1\\


%------------------------------------------------------------------------------------------------------------------

\hrulefill   

\begin{table}[!ht]
\begin{center}
\begin{tabularx}{\textwidth}{ X X}
\hspace{0.5cm} CompleteTies & \textit{Implements complete case methodology }\\
\end{tabularx}
\end{center}
\end{table} 

\vspace{-0.5cm}

\hrulefill  

\vspace{0.5cm}

\subsection*{Description}

Calculates the probability distribution using the complete case methodology, that is, using only the patterns formed without repeated elements.

\vspace{0.5cm}

\subsection*{Usage}

\begin{lstlisting}
   completeTies(elements, pattern)
\end{lstlisting}

\vspace{0.5cm}

\newpage

\subsection*{Arguments}

\begin{table}[!ht]
\begin{center}
\begin{tabularx}{\textwidth}{X X}
\hspace{0.5cm} \textit{elements} & Sub-sets of elements of the time series with dimension d of delay t.\\
\hspace{0.5cm} \textit{pattern} & Ordinal patterns formed from the time series with dimension d of delay t.\\
\end{tabularx}
\end{center}
\end{table} 

\subsection*{Examples}

\begin{lstlisting}
 set.seed(1234567890, kind = "Mersenne-Twister")
 x = runif(110000)
 d = 3
 del = 1
 elem = formationPattern(x = x, dimension = d, delay = del,
 			option = 0)
 pat = formationPattern(x = x, dimension = d, delay = del, 
 			option = 1)
 completeTies(elements, pattern)
\end{lstlisting}


%------------------------------------------------------------------------------------------------------------------

\hrulefill   

\begin{table}[!ht]
\begin{center}
\begin{tabularx}{\textwidth}{ X X}
\hspace{0.5cm} dataDriven & \textit{Calculates Data-Driven Imputation of a Time Series}\\
\end{tabularx}
\end{center}
\end{table} 

\vspace{-0.5cm}

\hrulefill  

\vspace{0.5cm}

\subsection*{Description}

Returns the numeric vector supplied with the given decimal number.

\vspace{0.5cm}

\subsection*{Usage}

\begin{lstlisting}
   dataDriven(x = x, dimension = d, delay = del)
\end{lstlisting}

\vspace{0.5cm}

\subsection*{Arguments}

\begin{table}[!ht]
\begin{center}
\begin{tabularx}{\textwidth}{X X}
\hspace{0.5cm} \textit{x} & A numeric vector (e.g. a time series)\\
\hspace{0.5cm} \textit{dimension} & Dimension size of ordinal patterns\\
\hspace{0.5cm} \textit{delay} & Size of the delay of ordinal patterns\\
\end{tabularx}
\end{center}
\end{table} 

\subsection*{Examples}

\begin{lstlisting}
 set.seed(1234567890, kind = "Mersenne-Twister")
 x <- runif(110000)
 d <- 3
 del <- 1
 dataDriven(x = x, dimension = d, delay = del)
\end{lstlisting}

\vspace{0.5cm}

%------------------------------------------------------------------------------------------------------------------

\hrulefill   

\begin{table}[!ht]
\begin{center}
\begin{tabularx}{\textwidth}{ X X}
\hspace{0.5cm} definePatterns & \textit{Calculates all ordinal patterns with a given dimension}\\
\end{tabularx}
\end{center}
\end{table} 

\vspace{-0.5cm}

\hrulefill  

\vspace{0.5cm}

\subsection*{Description}

Returns the patterns with dimension d.

\vspace{0.5cm}

\subsection*{Usage}

\begin{lstlisting}
   definePatterns(d = d)
\end{lstlisting}

\vspace{0.5cm}

\subsection*{Arguments}

\begin{table}[!ht]
\begin{center}
\begin{tabularx}{\textwidth}{X X}
\hspace{0.5cm} \textit{d} & Dimension size of ordinal patterns\\
\end{tabularx}
\end{center}
\end{table} 

\subsection*{Examples}

\begin{lstlisting}
 d <- 3
 definePatterns(d = d)
\end{lstlisting}

\vspace{0.5cm}

\newpage

%------------------------------------------------------------------------------------------------------------------

\hrulefill   

\begin{table}[!ht]
\begin{center}
\begin{tabularx}{\textwidth}{ X X}
\hspace{0.5cm} equalitiesValues & \textit{Calculates the percentage of repeated elements in a time series}\\
\end{tabularx}
\end{center}
\end{table} 

\vspace{-0.5cm}

\hrulefill  

\vspace{0.5cm}

\subsection*{Description}

Analyzes and calculates the percentage of how many repeated elements a time series has.

\vspace{0.5cm}

\subsection*{Usage}

\begin{lstlisting}
   equalitiesValues(x = x)
\end{lstlisting}

\vspace{0.5cm}

\subsection*{Arguments}

\begin{table}[!ht]
\begin{center}
\begin{tabularx}{\textwidth}{X X}
\hspace{0.5cm} \textit{x} & A numeric vector (e.g. a time series)\\
\end{tabularx}
\end{center}
\end{table} 

\subsection*{Examples}

\begin{lstlisting}
 set.seed(1234567890, kind = "Mersenne-Twister")
 x <- runif(110000)
 equalitiesValues(x = x)
\end{lstlisting}

\vspace{0.5cm}

%------------------------------------------------------------------------------------------------------------------

\hrulefill   

\begin{table}[!ht]
\begin{center}
\begin{tabularx}{\textwidth}{ X X}
\hspace{0.5cm} formationPattern & \textit{}\\
\end{tabularx}
\end{center}
\end{table}

\vspace{-0.5cm}

\hrulefill  

\vspace{0.5cm}

\subsection*{Description}

Divides the time series into subsets with a certain size and delay and calculates its ordinal patterns.

\vspace{0.5cm}

\subsection*{Usage}

\begin{lstlisting}
 formationPattern(serie = x, dimension = d,delay = del,
 			option = opt)
\end{lstlisting}

\vspace{0.5cm}

\subsection*{Arguments}

\begin{table}[!ht]
\begin{center}
\begin{tabularx}{\textwidth}{X X}
\hspace{0.5cm} \textit{serie} & A numeric vector (e.g. a time series)\\ \\
\hspace{0.5cm} \textit{dimension} & Dimension size of ordinal patterns\\ \\
\hspace{0.5cm} \textit{delay} & Size of the delay of ordinal patterns\\ \\
\hspace{0.5cm} \textit{option} & Determines the data set to be returned. The parameter must be 0 for ordinal patterns formed or different from 0 to return the subsets of formed elements.\\ \\
\end{tabularx}
\end{center}
\end{table} 

\subsection*{Examples}

\begin{lstlisting}
 set.seed(1234567890, kind = "Mersenne-Twister")
 x <- runif(110000)
 d <- 3
 del <- 1
 opt <- 0
 formationPattern(serie = x, dimension = d,delay = del,
 			option = opt)
\end{lstlisting}

\vspace{0.5cm}

%------------------------------------------------------------------------------------------------------------------

\hrulefill   

\begin{table}[!ht]
\begin{center}
\begin{tabularx}{\textwidth}{ X X}
\hspace{0.5cm} myPermute & \textit{Calculates all possible combinations of a set by exchanging only a subset of elements}\\
\end{tabularx}
\end{center}
\end{table} 

\vspace{-0.5cm}

\hrulefill  

\vspace{0.5cm}

\subsection*{Description}

Returns a numeric array containing all possible permutations of a vector by varying only a given subset.

\vspace{0.5cm}

\subsection*{Usage}

\begin{lstlisting}
    mypermute(vector = x, init = i, end = e)
\end{lstlisting}

\vspace{0.5cm}

\subsection*{Arguments}

\begin{table}[!ht]
\begin{center}
\begin{tabularx}{\textwidth}{X X}
\hspace{0.5cm} \textit{vector} & A numeric vector\\
\hspace{0.5cm} \textit{init} & Beginning of subset\\
\hspace{0.5cm} \textit{end} & End of subset\\
\end{tabularx}
\end{center}
\end{table} 

\subsection*{Examples}

\begin{lstlisting}

 x <- c(1:10)
 i <- 2
 e <- 6
 mypermute(vector = x, init = i, end = e)
   
\end{lstlisting}

\vspace{0.5cm}

%------------------------------------------------------------------------------------------------------------------

\hrulefill   

\begin{table}[!ht]
\begin{center}
\begin{tabularx}{\textwidth}{ X X}
\hspace{0.5cm} precision & \textit{Determines a precise number of a set of numbers}\\
\end{tabularx}
\end{center}
\end{table} 

\vspace{-0.5cm}

\hrulefill  

\vspace{0.5cm}

\subsection*{Description}

Returns the numeric vector supplied with the given decimal number.

\vspace{0.5cm}

\subsection*{Usage}

\begin{lstlisting}
   precision(x = x, y = y)
\end{lstlisting}

\vspace{0.5cm}

\newpage

\subsection*{Arguments}

\begin{table}[!ht]
\begin{center}
\begin{tabularx}{\textwidth}{X X}
\hspace{0.5cm} \textit{x} & A numeric vector (e.g. a time series)\\
\hspace{0.5cm} \textit{y} & Number of decimal places\\
\end{tabularx}
\end{center}
\end{table} 

\subsection*{Examples}

\begin{lstlisting}
   
 set.seed(1234567890, kind = "Mersenne-Twister")
 x <- runif(110000)
 y <- 4
 precision(x = x, y = y)

\end{lstlisting}

\vspace{0.5cm}

%------------------------------------------------------------------------------------------------------------------

\hrulefill   

\begin{table}[!ht]
\begin{center}
\begin{tabularx}{\textwidth}{ X X}
\hspace{0.5cm} shannonNormalize & \textit{Calculates the normalized Shannon entropy}\\
\end{tabularx}
\end{center}
\end{table} 

\vspace{-0.5cm}

\hrulefill  

\vspace{0.5cm}

\subsection*{Description}

Returns the normalized Shannon entropy of a probability distribution.

\vspace{0.5cm}

\subsection*{Usage}

\begin{lstlisting}
   shannonNormalized(p = prob)
\end{lstlisting}

\vspace{0.5cm}

\subsection*{Arguments}

\begin{table}[!ht]
\begin{center}
\begin{tabularx}{\textwidth}{X X}
\hspace{0.5cm} \textit{p} & A numerical vector containing an ordinal pattern distribution\\
\end{tabularx}
\end{center}
\end{table} 

\subsection*{Examples}

\begin{lstlisting}
   
 set.seed(1234567890, kind = "Mersenne-Twister")
 x <- runif(110000)
 d <- 3
 del <- 1
 prob = dataDriven(x = x, dimension = d, delay = del)
 shannonNormalized(p = prob)

\end{lstlisting}

\vspace{0.5cm}

\end{document}
